\documentclass[letterpaper]{resume}
\usepackage{paralist}

\author{Gregory Wilbur}
    
\email{gwilbur@gmail.com}
\website{http://wil.bz/}
\phone{978-269-4851}

\streetaddress{200 Henry St.}
\apartmentnum{Apt. 1603}
\citystatezip{Stamford, CT  06902}

\begin{document}
\maketitle

\section{Software Engineering Qualifications}
\vspace{\secskip}

\begin{compactitem}
	\item Proficiency in C/C++, Java, Python; experience with Erlang, Lua, Ruby, Scheme/Lisp, Unix shell scripts, x86 Assembly. \par
	\item Experience understanding and modifying large, many-file codebases including Linux and GCC.\par
	\item Ability working in situations with little supervision, as well as situations with extensive team interaction. \par
\end{compactitem}

\section{Software Engineering Experience}

\affiliation
{Software Engineer}
{FactSet Research Systems, Inc.}
{July 2010-Present}
{Norwalk, CT}

\begin{compactitem}
	\item Developed and maintained several systems for processing raw financial data and loading them into databases. \par
	\item Maintenance duties included responding to pages outside of work hours \par
\end{compactitem}

\affiliation
{Research Assistant}
{University of Rochester Department of Computer Science}
{June 2009-May 2010}
{Rochester, NY}

\begin{compactitem}
	\item Developed and studied a parallel discrete event simulator on top of the Rochester Software Transactional Memory system. \par
\end{compactitem}

\affiliation
{Assistant Unix System Administrator}
{Tufts University}
{July 2008-August 2008}
{Medford, MA}

\begin{compactitem}
	\item Installed and maintained Unix servers and desktops for use by multiple departments on campus. \par
	\item Wrote documentation on installation and use of several applications including Erlang, Linux, VMware, and GNUpg. \par
\end{compactitem}

\affiliation
{Research Assistant}
{Tufts University Department of Chemistry}
{June 2007-August 2007}
{Medford, MA}

\begin{compactitem}
	\item Worked with the designers of the MECA (Microscopy, Electrochemistry and Conductivity Analyzer) module for NASA's Phoenix mission to Mars in search of signs of past or potential habitability. \par
	\item Developed software in LabVIEW controlling the atmosphere and temperature in chamber that simulated conditions on the surface of Mars. \par
\end{compactitem}

\affiliation
{Quality Control Assistant}
{Cambridge Isotope Laboratories}
{June 2006-August 2006}
{Andover, MA}

\begin{compactitem}
       \item Used Java to develop custom Laboratory Information Management System to organize and track analytical data for a chemical inventory containing over 8,000 products. \par
\end{compactitem}

\section{Teaching Experience}

\affiliation
{Computer Science and Mathematics Teaching Assistant and Tutor}
{University of Rochester}
{January 2007-Present}
{Rochester, NY}

\begin{compactitem}
	\item Taught students both one-on-one as well as in groups of approximately 10. \par
	\item Explained technical concepts to many undergruaduates with non-technical backgrounds. \par
\end{compactitem}

\section{Computer Science and Mathematics Education}

\affiliation{University of Rochester}
            {Bachelor of Science in Computer Science and Bachelor of Science in Mathematics}
			{Rochester, NY}
			{Expected May 2010}

\begin{compactitem}
	\item Cumulative GPA: 3.30 / 4.00 \par
\end{compactitem}

\section{Computer Science Courses and Projects}
\vspace{\secskip}

\begin{compactitem}
	\item \textbf{Computer Networking:} \textit{(Fall 2009)} Wrote implementations of several networking protocols, including basic HTTP proxy server, distance vector routing protocol, and Gnutella-like peer-to-peer searching algorithm. For term project, developed multi-player networked version of Asteroids. \par
	
	\item \textbf{Parallel and Distributed Systems:} \textit{(Spring 2009)} Wrote parallel applications using several parallel or distributed API's including Pthreads, MPI, Cilk, and OpenMP. \par

	\item \textbf{Advanced Programming Systems:} \textit{(Spring 2009)} Group modified the GNU Compiler Collection to implement own compiler optimizations as part of competition with other groups in class. \par

	\item \textbf{Undergraduate Problem Seminar:} \textit{(Spring 2009)} Discussed several open-ended problems and wrote academic papers summarizing findings. \par
	
	\item \textbf{Operating Systems:} \textit{(Fall 2008)} Modified Linux kernel (version 2.6) to implement algorithms for several tasks, including process scheduling and virtual memory management. \par
	
	\item \textbf{Artificial Intelligence Research: Quagents:} \textit{(Spring 2007)} Modified Quake 3 game engine for students in the department's Artificial Intelligence class to use for the visualization of agents they develop. \par
\end{compactitem}

\end{document}
